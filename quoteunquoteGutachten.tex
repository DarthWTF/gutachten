\documentclass[a4paper,ngerman, 14pt] {scrartcl}
\usepackage[T1]{fontenc}
\usepackage[utf8]{inputenc}
\usepackage[ngerman]{babel}
\usepackage{blindtext}
\usepackage{csquotes}
\usepackage{hyperref}
\usepackage{graphicx}
\usepackage{caption}
\usepackage{setspace}
\usepackage{MnSymbol}
\usepackage[top=2cm, bottom=3.5cm, left=3cm, right=2.5cm]{geometry} 

\begin{document}
\section{Einleitung}
Der AsTA hat vor einiger Zeit den Entschluss gefasst ein Pedelec Lastenrad zu erwerben um es als Leihrad für die Studierenden anzubieten. Es soll das mittlerweile sehr angejahrte und mittlerweile nicht mehr fahrtüchtige Lastenrad im Bestand ersetzen. Ziel ist im Bereich Lastentransport den Studierenden Alternativen zum Auto zu bieten und die neue Mobilität zu fördern.\\
Zu diesem Zweck ist Erik Wohlfeil, Nachhaltigkeitsbeauftragter des Studierendenparlaments an uns, den Arbeitskreis Fahrradcampus, herangetreten mit der bitte um technische Unterstützung bei der Anschaffungsentscheidung. Eine frühere Kaufempfehlung existierte bereits, diese kann aber angesichts der mittlerweile geänderten Marktlage als nichtig erachtet werden.\\
\section{Anforderungen die das Rad stellt}
Ein hochwertiges modernes Fahrrad, stellt andere Anforderungen an den Nutzer als man es vielleicht von älteren Rädern gewohnt ist. Langrfristige Nutzung eines Lastenpedelecs stellt folgende Anforderungen:\\
\begin{itemize}
    \item \textbf{Regelmäßige Wartung bei einer Fachwerkstatt.} Am besten ist es eine feste Vereinbarung mit einem Fachhändler einzugehen, idealerweise der Selbe bei dem das Rad erworben wurde
    \item \textbf{Schutz vor Witterung.} Ein Rad und ein Pedelec im besonderen dürfen bei Nichtnutzung nicht einfach im Freien stehen, wie es mit dem alten Lastenrad Usus war. Eine Möglichkeit das Rad unterzustellen muss vorhanden sein. Schutz vor Witterung ist das Wichtigste, Schutz vor Kälte ist erstrebenswert aber nicht zwingend notwendig.
    \item \textbf{Kurze Unterrichtung von Neunutzer:innen.} Die Technik eines Pedelecs und das andere Fahrverhalten eines Lastenrad machen es wichtig das Nutzer die nicht mit diesen Dingen vertraut sind eine kurze Einweisung erhalten. Jede Person die den Lastenradverleih durchführt sollte diese kurze Einweisung geben können und ggfs. eine kurze Anpassung an den Nutzer durchführen.
\end{itemize}
All diese Punkte müssen vor der Anschaffung geklärt sein.
\section{Grundvorraussetzungen an das Rad}
Bevor wir in die engere Auswahl gegangen sind haben wir einen Grundrahmen formuliert in den alle Räder passen müssen. Folgende Punkte kamen zusammen:\\
\begin{itemize}
    \item \textbf{Long John Rahmenform.} BILD HIER EINFÜGEN\\ Lastenräder gibt es in vielen unterschiedlichen Konfigurationen. Von Anfang an war klar dass nur Zweiräder in Frage kommen da sich Dreirädern bei Kurvenfahrten unvorhersehbar verhalten. Die Long John Konfiguration, bei der die Ladefläche Tief vor dem Fahrer liegt, ermöglicht es die Länge des Rades verlässlich während der Fahrt einzuschätzen und liefert durch den tiefen Schwerpunkt ein sicheres Fahrverhalten.
    \item \textbf{Mittelmotor.} Motorisierung eines Fahrrads kann an drei Punkten erfolgen. In der Vorderradnabe, an der Kurbel oder in der Hinterradnabe. Frontmotoren liefern in unseren Augen kein zufriedenstellendes Unterstützungsverhalten. Heckmotoren tun dies, jedoch neigen sie in der Hand von weniger erfahrenen Nutzern zum überhitzen unter Last. Mittelmotoren haben den größten Marktanteil unter den Pedelecs, sie liefern verlässliche und kontrollierbare Unterstützung, sowie den höchsten Reifegrad unter den Antriebssystemen.
    \item \textbf{Hydraulische Scheibenbremsen.} Hohes Gewicht des Fahrrads und elektrische Unterstützung erfordern angemessene Bremsen. Manche Hersteller setzen an diesem Punkt leider manchmal den Rotstift an. Hydraulische Scheibenbremsen sind unserer Meinung nach das Mindeste, idealerweise mit großen Bremsscheiben oder sogar Vierkolbenbremskörpern.
    \item \textbf{Nabenschaltung.} Das Lastenrad wird in den Händen der Studierenden \textbf{nicht zimperlich} behandelt werden. Wichtig ist daher intuitive Bedienbarkeit und Robustheit gegenüber Fehlbedienung. Boshaft formuliert: Es muss idiotensicher sein. In diesem Anwendungsfall bieten sich gekapselte Nabenschaltungen gegenüber einer Kettenschaltung an, hier gibt es keine exponierten Teile und das Risiko einer Beschädigung durch Missbrauch ist deutlich geringer.
    \item \textbf{Anpassbarkeit.} Lastenräder kommen nahezu ohne Ausnahme in genau einer Größe. Aus diesem Grund treffen die meisten Hersteller Maßnahmen um das Rad einem möglichst großen Spektrum an Fahrer:innen nutzbar zu machen. Die Wirksamkeit dieser Maßnahmen variiert zwischen Modellen.
\end{itemize}
\section{Tests}
Unerlässlich für eine Entscheidungsfindung sind Praxistests, diese fanden bis jetzt in zwei Formen statt. Einer der Autoren ist in einem Radladen tätig und hat dadurch Berührungspunkte mit Lastenrädern, weiterhin wurde der Halt der Cargobike Roadshow in Böblingen am 20.09.2020 genutzt um weitere Räder zu testen.
\begin{itemize}
    \item \textbf{Bakfiets Classic Long Steps SS8.} Ein Bakfiets mit Modell wird in besagten Radladen als Ausstellungsobjekt genutzt sowie zum Transport von Kindern, Getränken und manchmal auch anderen Fahrrädern. Das Bakfiets kann leider nicht auf ganzer Linie überzeugen. Fahrstabilität ist unterer Durchschnitt und Anpassbarkeit gut, in den Details fallen die Dinge jedoch auseinander. Im Hinblick auf Wartung ist das Rad eine Zumutung, Schwächen in Verarbeitung und Konstruktion trüben das Bild weiter. Die Bedienung der Parkstütze ist hakelig.\\ Mit 4030€ ist das Rad vergleichsweise preiswert.
    \item \textbf{Ca Go FS 200.} Beim Testevent in Böblingen stach das Ca Go heraus. Eine brandneue Entwicklung, vollgestopft mit state of the art Technik. Feinste Komponenten, integrierte Displays, hydraulisch unterstützter Deckel für die Ladebox und mehr. Die Kehrseite des High-Tech ist dass wir es für ein wenig zu sensibel für den Verleih erachten. Weiterhin war die Lenkung schwammig, ein schwerlich akzeptabler Kompromiss für ca 6500€.
    \item \textbf{Urban Arrow Family.}
\end{itemize}
\end{document}